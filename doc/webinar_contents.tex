%%%%%%%%%%%%%%%%%%%%%%%%%%%%%%%%%%%%%%%%%%%%%%%%%%%%%%%%%%%%%%%
%%%%%%%%%%%%%%%%%%%%%%%%%%%%%%%%%%%%%%%%%%%%%%%%%% Introduction

\section{Introduction}
\begin{frame}
    \frametitle{Introduction}
    \vspace{5mm}
    
    \begin{itemize}
      \item 
    \end{itemize}
\end{frame}

\subsection{Philosophy}
\begin{frame}
    \frametitle{Philosophy}
    \vspace{5mm}
    
    \begin{itemize}
      \item 
    \end{itemize}
\end{frame}

\subsection{Characteristics}
\begin{frame}
    \frametitle{Characteristics}
    \vspace{5mm}
    
    \begin{itemize}
      \descitem{Interpreted}
      \descitem{Garbage Collected}
      \descitem{Dynamically Typed}
      \descitem{High-Level}
      \descitem{Late Binding}
      \descitem{}
    \end{itemize}
\end{frame}

\subsection{Positioning}
\begin{frame}
    \frametitle{Positioning}
    \vspace{5mm}
    
    \begin{itemize}
      \item 
    \end{itemize}
\end{frame}

\subsection{Good Matches}
\begin{frame}
    \frametitle{Good Matches}
    \vspace{5mm}
    
    \begin{itemize}
      \item 
    \end{itemize}
\end{frame}

\subsection{Bad Matches}
\begin{frame}
    \frametitle{Bad Matches}
    \vspace{5mm}
    
    \begin{itemize}
      \item 
    \end{itemize}
\end{frame}

%%%%%%%%%%%%%%%%%%%%%%%%%%%%%%%%%%%%%%%%%%%%%%%%%%%%%%%%%%%%%%%
%%%%%%%%%%%%%%%%%%%%%%%%%%%%%%%%%%%%%%%%%%%%%%% Getting Started

\section{Getting Started}
\begin{frame}
    \vspace{25mm}
    \begin{center}
        \Huge{Part 1:\\Getting Started}
    \end{center}
\end{frame}

%%%%%%%%%%%%%%%%%%%%%%%%%%%%%%%%%%%%%%%%%%%%%%%%%%%%%%%%%%%%%%%
%%%%%%%%%%%%%%%%%%%%%%%%%%%%%%% Basic Datatypes and -Structures

\section{Basic Datatypes and -Structures}
\begin{frame}
    \vspace{25mm}
    \begin{center}
        \Huge{Part 2:\\Basic Datatypes and -Structures}
    \end{center}
\end{frame}

\subsection{Boolean Operations}
\begin{frame}
    \frametitle{Boolean Operations}
    \vspace{5mm}
    
    \begin{itemize}
      \item
    \end{itemize}
\end{frame}

\defverbatim[colored]\contentTypeIntrospectionI{
\begin{minted}{pycon}
>>> t = type(True)
>>> t
<class 'bool'>
>>> type(t)
<class 'type'>
>>> type(bool)
<class 'type'>
>>> t == bool
True
\end{minted}
}
\defverbatim[colored]\contentTypeIntrospectionII{
\begin{minted}{pycon}
>>> def fun(var): return var
... 
>>> type(fun)
<class 'function'>
>>> f = fun
>>> type(f)
<class 'function'>
>>> f(1)
1
>>> g = lambda a: a
>>> type(g)
<class 'function'>
>>> g(1)
1
\end{minted}
}
\subsection{Type Introspection}
\begin{frame}
    \frametitle{Type Introspection}
    \vspace{0mm}
    \contentTypeIntrospectionI
\end{frame}
\begin{frame}
    \frametitle{Type Introspection}
    \vspace{0mm}
    \contentTypeIntrospectionII
\end{frame}

\subsection{Object-Orientation}
\begin{frame}
    \frametitle{Object-Orientation}
    \vspace{5mm}
    
    \begin{itemize}
      \item self, late binding
    \end{itemize}
\end{frame}

%%%%%%%%%%%%%%%%%%%%%%%%%%%%%%%%%%%%%%%%%%%%%%%%%%%%%%%%%%%%%%%
%%%%%%%%%%%%%%%%%%%%%%%%%%%%%%%%%%%%%%%%%%%%%%%%%% Flow Control

\section{Flow Control}
\begin{frame}
    \vspace{25mm}
    \begin{center}
        \Huge{Part 3:\\Flow Control}
    \end{center}
\end{frame}

\subsection{Missing For-Loop}
\defverbatim[colored]\contentForI{
\begin{minted}{python}
for line in lines:
    print(line)
\end{minted}
}
\defverbatim[colored]\contentForII{
\begin{minted}{python}
for i in range(len(lines)):
    line = lines[i]
    print(str(i)+': '+line)
\end{minted}
}
\begin{frame}
    \frametitle{Missing For-Loop}
    \vspace{3mm}
    Python does not have a \textsl{for} loop.
    
    \pause
    \vspace{3mm}
    Python has a \textsl{foreach} loop.
    \pause
    
    \vspace{6mm}
    Iterating over a list:
    \vspace{3mm}
    \contentForI
    \pause
    
    \vspace{6mm}
    Iterating over a list with access to the index:
    \vspace{3mm}
    \contentForII
\end{frame}

\subsection{Generating Ranges of Integers}
\defverbatim[colored]\contentRange{
\begin{minted}{pycon}
>>> range(5)
range(0, 5)
>>> list(range(5))
[0, 1, 2, 3, 4]
>>> list(range(1,5))
[1, 2, 3, 4]
>>> list(range(1,5,2))
[1, 3]
>>> for i in range(1,5,2):
...     print(i)
... 
1
3
\end{minted}
}
\begin{frame}
    \frametitle{Generating Ranges of Integers}
    \vspace{3mm}
    The \texttt{range} function returns a generator for a sequence of integers.
    \pause
    \vspace{3mm}
    \contentRange
\end{frame}

\subsection{Branching}
\begin{frame}
    \frametitle{Branching}
    \vspace{5mm}
    
    \begin{itemize}
      \item 
    \end{itemize}
\end{frame}

%%%%%%%%%%%%%%%%%%%%%%%%%%%%%%%%%%%%%%%%%%%%%%%%%%%%%%%%%%%%%%%
%%%%%%%%%%%%%%%%%%%%%%%%%%%%%%%%%%%%%%%%%%%%%%%%%%%%%%% Strings

\section{Strings}
\begin{frame}
    \vspace{25mm}
    \begin{center}
        \Huge{Part 4:\\Strings}
    \end{center}
\end{frame}

\subsection{Basic Operations}
\defverbatim[colored]\contentStrings{
\begin{minted}{pycon}
>>> initial = ' once upon a time  '
>>> initial
' once upon a time  '
>>> len(initial)
19
>>> stripped = initial.strip()
>>> stripped
'once upon a time'
>>> words = stripped.split(' ')
>>> words
['once', 'upon', 'a', 'time']
>>> words[1], stripped[1]
('upon', 'n')
>>> joined = '_'.join(words)
>>> joined
'once_upon_a_time'
\end{minted}
}
\begin{frame}
    \frametitle{\textbf{Strings} Basic Operations}
    \vspace{0mm}
    \contentStrings
\end{frame}

%%%%%%%%%%%%%%%%%%%%%%%%%%%%%%%%%%%%%%%%%%%%%%%%%%%%%%%%%%%%%%%
%%%%%%%%%%%%%%%%%%%%%%%%%%%%%%%%%%%%%%%%%%%%%%%%%%%%%%%%% Files

\section{Files}
\begin{frame}
    \vspace{25mm}
    \begin{center}
        \Huge{Part 5:\\Files}
    \end{center}
\end{frame}

\defverbatim[colored]\contentFilesI{
\begin{minted}{python}
fo = open(filename, mode)
# do something with 'fo'
fo.close()
\end{minted}
}
\defverbatim[colored]\contentFilesIr{
\begin{minted}{python}
lines = fo.readlines()
\end{minted}
}
\defverbatim[colored]\contentFilesIw{
\begin{minted}{python}
fo.writelines(lines)
\end{minted}
}
\begin{frame}
    \frametitle{\textbf{Files} Basics}
    \vspace{0mm}
    \contentFilesI
    \pause
    \vspace{3mm}
    Modes:
    \begin{itemize}
        \item \texttt{'r'} read mode \textbf{(default $\Rightarrow$ may be omitted)}
        \item \texttt{'w'} write mode (clears file on open)
        \item \texttt{'a'} append mode (continues at end of file)
    \end{itemize}
    \pause
    \vspace{3mm}
    Read using:
    \contentFilesIr
    \pause
    \vspace{3mm}
    Write using:
    \contentFilesIw
\end{frame}

\defverbatim[colored]\contentFilesII{
\begin{minted}{python}
with open(input_filename) as fo:
    lines = fo.readlines()

with open(output_filename, 'w') as fo:
    fo.writelines(lines)
\end{minted}
}
\begin{frame}
    \frametitle{\textbf{Files} Implicit Closing of File Objects}
    \vspace{0mm}
    \contentFilesII
\end{frame}

\defverbatim[colored]\contentFilesIII{
\begin{minted}{python}
with open(filename) as fo:
    for line in fo:
        print(line)
\end{minted}
}
\begin{frame}
    \frametitle{\textbf{Files} Reading from a Stream of Lines}
    \vspace{0mm}
    \contentFilesIII
\end{frame}

%%%%%%%%%%%%%%%%%%%%%%%%%%%%%%%%%%%%%%%%%%%%%%%%%%%%%%%%%%%%%%%
%%%%%%%%%%%%%%%%%%%%%%%%%%%%%%%%%%%%%%%%%% Marshalling and JSON

\section{Marshalling and JSON}
\begin{frame}
    \vspace{25mm}
    \begin{center}
        \Huge{Part 6:\\Marshalling and JSON}
    \end{center}
\end{frame}

%%%%%%%%%%%%%%%%%%%%%%%%%%%%%%%%%%%%%%%%%%%%%%%%%%%%%%%%%%%%%%%
%%%%%%%%%%%%%%%%%%%%%%%%%%%%%%%%%%%%%%%% Higher-Order Functions

\section{Higher-Order Functions}
\begin{frame}
    \vspace{25mm}
    \begin{center}
        \Huge{Part 7:\\Higher-Order Functions}
    \end{center}
\end{frame}

%%%%%%%%%%%%%%%%%%%%%%%%%%%%%%%%%%%%%%%%%%%%%%%%%%%%%%%%%%%%%%%
%%%%%%%%%%%%%%%%%%%%%%%%%%%%%%%%%%%%%%%%%%%% Working in Modules

\section{Working in Modules}
\begin{frame}
    \vspace{25mm}
    \begin{center}
        \Huge{Part 8:\\Working in Modules}
    \end{center}
\end{frame}

%%%%%%%%%%%%%%%%%%%%%%%%%%%%%%%%%%%%%%%%%%%%%%%%%%%%%%%%%%%%%%%
%%%%%%%%%%%%%%%%%%%%%%%%%%%%%%%%%%%%%%%%%%%%%%%%%%%%%% Pitfalls

\section{Pitfalls}
\begin{frame}
    \vspace{25mm}
    \begin{center}
        \Huge{Part 9:\\Pitfalls}
    \end{center}
\end{frame}

\subsection{Mutable Default Arguments}
\begin{frame}
    \frametitle{Mutable Default Arguments}
    \vspace{5mm}
    
    \begin{itemize}
      \item 
    \end{itemize}
\end{frame}

\subsection{Scope and Globals}
\begin{frame}
    \frametitle{Scope and Globals}
    \vspace{5mm}
    
    \begin{itemize}
      \item 
    \end{itemize}
\end{frame}

%%%%%%%%%%%%%%%%%%%%%%%%%%%%%%%%%%%%%%%%%%%%%%%%%%%%%%%%%%%%%%%
%%%%%%%%%%%%%%%%%%%%%%%%%%%%%%%%%%%%%%%%%%%%%%%%%%%% Next Steps

\section{Next Steps}
\begin{frame}
    \vspace{25mm}
    \begin{center}
        \Huge{Part 10:\\Next Steps}
    \end{center}
\end{frame}

\subsection{SymPy}
\begin{frame}
    \frametitle{SymPy}
    \vspace{5mm}
    
    \begin{itemize}
      \item 
    \end{itemize}
\end{frame}

\subsection{NumPy}
\begin{frame}
    \frametitle{NumPy}
    \vspace{5mm}
    
    \begin{itemize}
      \item 
    \end{itemize}
\end{frame}

\subsection{Pandas}
\begin{frame}
    \frametitle{Pandas}
    \vspace{3mm}
    \hspace{105mm}\includegraphics[width=35mm]{../src/running-average.pdf}
    \vspace{-73mm}\inputminted[fontsize=\footnotesize]{python}{../src/pandas-example.py}
\end{frame}

\subsection{AsyncIO}
\begin{frame}
    \frametitle{AsyncIO}
    \vspace{5mm}
    
    \begin{itemize}
      \item 
    \end{itemize}
\end{frame}

\subsection{BeautifulSoup}
\begin{frame}
    \frametitle{BeautifulSoup}
    \vspace{5mm}
    
    \begin{itemize}
      \item 
    \end{itemize}
\end{frame}

\subsection{Managing Modules}
\begin{frame}
    \frametitle{Managing Modules}
    \vspace{5mm}
    
    \begin{itemize}
      \item 
    \end{itemize}
\end{frame}

\subsection{Other Resources}
\begin{frame}
    \frametitle{NumPy}
    \vspace{5mm}
    
    \begin{itemize}
      \item 
    \end{itemize}
\end{frame}

%%%%%%%%%%%%%%%%%%%%%%%%%%%%%%%%%%%%%%%%%%%%%%%%%%%%%%%%%%%%%%%
%%%%%%%%%%%%%%%%%%%%%%%%%%%%%%%%%%%%%%%%%%%%%%%%%%%%% Questions

\section{Questions}
\begin{frame}
    \frametitle{\textbf{Questions and Comments?}}
    \vspace{-15mm}
    \begin{center}
    \includegraphics[scale=0.4]{./figs/Boy-asking-question.pdf}
    \end{center}
    \vspace{-25mm}
    \scalebox{0.2}{\url{https://openclipart.org/detail/238687/boy-thinking-of-question}}
\end{frame}

